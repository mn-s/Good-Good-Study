\documentclass[UTF8]{ctexart}
\usepackage{amsmath} %导入amsmath宏包支持数学公式
\title{Hello,World}
\author{Quanyin Tang}
\date{2018年1月17日}
\begin{document}

\maketitle

Einstein's $E=mc^2$. %行间  $...$

\[ E=mc^2 \]  %行间 \[ ... \]

\begin{equation} %编辑行间公式
E=mc^2
\end{equation}

\[C^m_n\]		% 在数学模式中,需要表示上标,可以使用 ^ 来实现(下标则是 _)
\[ z = r\cdot e^{2\pi i}. \] %上下标默认只作用于之后的一个字符,如果想对连续的几个字符起作用,请将这些字符用花括号 {} 括起来

%根号,分号
$\sqrt{x}$, $\frac{1}{2}$.
\[ \sqrt{x^2+y^2}, \]
\[ \frac{1}{2}. \]
$\dfrac{1}{2}$ %强制行内模式的分式显示为行间模式的大小
\[ \tfrac{1}{2} \]%反之

%运算符
\[ \pm\; \times \; \div\; \cdot\; \cap\; \cup\; \geq\; \leq\; \neq\; \approx \; \equiv \] 

%连加、连乘、极限、积分
\[ \sum\; \prod\;  \lim\;  \int \] 
$ \sum_{i=1}^n i\quad \prod_{i=1}^n $
$ \sum\limits _{i=1}^n i\quad \prod\limits _{i=1}^n $ %用 \limits 和 \nolimits 来强制显式地指定是否压缩这些上下标
\[ \lim_{x\to0}x^2 \quad \int_a^b x^2 dx \]
\[ \lim\nolimits _{x\to0}x^2\quad \int\nolimits_a^b x^2 dx \]
\[ \iint\quad \iiint\quad \iiiint\quad \idotsint \] %多重积分

%% 括号等结果
\[ \Biggl(\biggl(\Bigl(\bigl((x)\bigr)\Bigr)\biggr)\Biggr) \] 
\[ \Biggl[\biggl[\Bigl[\bigl[[x]\bigr]\Bigr]\biggr]\Biggr] \]
\[ \Biggl \{\biggl \{\Bigl \{\bigl \{\{x\}\bigr \}\Bigr \}\biggr \}\Biggr\} \]
\[ \Biggl\langle\biggl\langle\Bigl\langle\bigl\langle\langle x
\rangle\bigr\rangle\Bigr\rangle\biggr\rangle\Biggr\rangle \]
\[ \Biggl\lvert\biggl\lvert\Bigl\lvert\bigl\lvert\lvert x
\rvert\bigr\rvert\Bigr\rvert\biggr\rvert\Biggr\rvert \]
\[ \Biggl\lVert\biggl\lVert\Bigl\lVert\bigl\lVert\lVert x
\rVert\bigr\rVert\Bigr\rVert\biggr\rVert\Biggr\rVert \]

%%省略号用\dots \cdots \vdots \ddots表示
\[ x_1,x_2,\dots ,x_n\quad 1,2,\cdots ,n\quad
\vdots\quad \ddots \]

%%矩阵
\[ 
\begin{pmatrix} a&b\\c&d \end{pmatrix} \quad
\begin{bmatrix} a&b\\c&d \end{bmatrix} \quad
\begin{Bmatrix} a&b\\c&d \end{Bmatrix} \quad
\begin{vmatrix} a&b\\c&d \end{vmatrix} \quad
\begin{Vmatrix} a&b\\c&d \end{Vmatrix} 
\]

%%行内小矩阵
Marry has a little matrix $( \begin{smallmatrix} a&b\\c&d \end{smallmatrix})$

%多行公式
%%长公式(不对齐),使用multline环境,不编号使用multline*代替
\begin{multline}
x=a+b+c+{}\\
d+e+f+g
\end{multline}
%%对齐,使用aligned
\[
\begin{aligned}
x={}& a+b+c+{}\\
&d+e+f+g
\end{aligned}
\]

%公式组,无需对齐用gather环境,对齐用align环境,不带编号的话加星号
\begin{gather}
a = b+c+d \\
x = y+z
\end{gather}
\begin{align}
a &= b+c+d \\
x &= y+z
\end{align}

%分段函数,用cases
\[
y=\begin{cases}
-x,\quad x\leq 0\\
x,\quad x>0
\end{cases}
\]

\end{document}

